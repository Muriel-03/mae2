\documentclass{article}

\usepackage[utf8]{inputenc}
\usepackage[spanish]{babel}
\usepackage{hyperref}
\usepackage{listings}
\lstset{
    inputencoding = utf8,  % Input encoding
    extendedchars = true,  % Extended ASCII
    literate      =        % Support additional characters
      {á}{{\'a}}1  {é}{{\'e}}1  {í}{{\'i}}1 {ó}{{\'o}}1  {ú}{{\'u}}1
      {Á}{{\'A}}1  {É}{{\'E}}1  {Í}{{\'I}}1 {Ó}{{\'O}}1  {Ú}{{\'U}}1
      {à}{{\`a}}1  {è}{{\`e}}1  {ì}{{\`i}}1 {ò}{{\`o}}1  {ù}{{\`u}}1
      {À}{{\`A}}1  {È}{{\`E}}1  {Ì}{{\`I}}1 {Ò}{{\`O}}1  {Ù}{{\`U}}1
      {ä}{{\"a}}1  {ë}{{\"e}}1  {ï}{{\"i}}1 {ö}{{\"o}}1  {ü}{{\"u}}1
      {Ä}{{\"A}}1  {Ë}{{\"E}}1  {Ï}{{\"I}}1 {Ö}{{\"O}}1  {Ü}{{\"U}}1
      {â}{{\^a}}1  {ê}{{\^e}}1  {î}{{\^i}}1 {ô}{{\^o}}1  {û}{{\^u}}1
      {Â}{{\^A}}1  {Ê}{{\^E}}1  {Î}{{\^I}}1 {Ô}{{\^O}}1  {Û}{{\^U}}1
      {œ}{{\oe}}1  {Œ}{{\OE}}1  {æ}{{\ae}}1 {Æ}{{\AE}}1  {ß}{{\ss}}1
      {ẞ}{{\SS}}1  {ç}{{\c{c}}}1 {Ç}{{\c{C}}}1 {ø}{{\o}}1  {Ø}{{\O}}1
      {å}{{\aa}}1  {Å}{{\AA}}1  {ã}{{\~a}}1  {õ}{{\~o}}1 {Ã}{{\~A}}1
      {Õ}{{\~O}}1  {ñ}{{\~n}}1  {Ñ}{{\~N}}1  {¿}{{?`}}1  {¡}{{!`}}1
      {°}{{\textdegree}}1 {º}{{\textordmasculine}}1 {ª}{{\textordfeminine}}1
      {£}{{\pounds}}1  {©}{{\copyright}}1  {®}{{\textregistered}}1
      {«}{{\guillemotleft}}1  {»}{{\guillemotright}}1  {Ð}{{\DH}}1  {ð}{{\dh}}1
      {Ý}{{\'Y}}1    {ý}{{\'y}}1    {Þ}{{\TH}}1    {þ}{{\th}}1    {Ă}{{\u{A}}}1
      {ă}{{\u{a}}}1  {Ą}{{\k{A}}}1  {ą}{{\k{a}}}1  {Ć}{{\'C}}1    {ć}{{\'c}}1
      {Č}{{\v{C}}}1  {č}{{\v{c}}}1  {Ď}{{\v{D}}}1  {ď}{{\v{d}}}1  {Đ}{{\DJ}}1
      {đ}{{\dj}}1    {Ė}{{\.{E}}}1  {ė}{{\.{e}}}1  {Ę}{{\k{E}}}1  {ę}{{\k{e}}}1
      {Ě}{{\v{E}}}1  {ě}{{\v{e}}}1  {Ğ}{{\u{G}}}1  {ğ}{{\u{g}}}1  {Ĩ}{{\~I}}1
      {ĩ}{{\~\i}}1   {Į}{{\k{I}}}1  {į}{{\k{i}}}1  {İ}{{\.{I}}}1  {ı}{{\i}}1
      {Ĺ}{{\'L}}1    {ĺ}{{\'l}}1    {Ľ}{{\v{L}}}1  {ľ}{{\v{l}}}1  {Ł}{{\L{}}}1
      {ł}{{\l{}}}1   {Ń}{{\'N}}1    {ń}{{\'n}}1    {Ň}{{\v{N}}}1  {ň}{{\v{n}}}1
      {Ő}{{\H{O}}}1  {ő}{{\H{o}}}1  {Ŕ}{{\'{R}}}1  {ŕ}{{\'{r}}}1  {Ř}{{\v{R}}}1
      {ř}{{\v{r}}}1  {Ś}{{\'S}}1    {ś}{{\'s}}1    {Ş}{{\c{S}}}1  {ş}{{\c{s}}}1
      {Š}{{\v{S}}}1  {š}{{\v{s}}}1  {Ť}{{\v{T}}}1  {ť}{{\v{t}}}1  {Ũ}{{\~U}}1
      {ũ}{{\~u}}1    {Ū}{{\={U}}}1  {ū}{{\={u}}}1  {Ů}{{\r{U}}}1  {ů}{{\r{u}}}1
      {Ű}{{\H{U}}}1  {ű}{{\H{u}}}1  {Ų}{{\k{U}}}1  {ų}{{\k{u}}}1  {Ź}{{\'Z}}1
      {ź}{{\'z}}1    {Ż}{{\.Z}}1    {ż}{{\.z}}1    {Ž}{{\v{Z}}}1
      % ¿ and ¡ are not correctly displayed if inconsolata font is used
      % together with the lstlisting environment. Consider typing code in
      % external files and using \lstinputlisting to display them instead.      
  }

\title{Detectando enfermedades en archivos de texto usando expresiones regulares}
\author{Fabián Villena}
\date{Junio 2025}

\begin{document}

\maketitle

Usted es científico de datos en un servicio de salud y se le pidió calcular la prevalencia de ciertas enfermedades en el repositorio de interconsultas no resueltas de la red de proveedores de salud. El equipo de tecnologías de información exportó el campo de hipótesis diagnóstica de cada una de las interconsultas en múltiples archivos de texto.

Este es un ejemplo del contenido de un archivo de texto en donde se menciona un conjunto de enfermedades:

\begin{lstlisting}[breaklines=true, extendedchars=true,numbers=left,frame=single]
- TRASTORNO DE LA REFRACCIÓN, NO ESPECIFICADO


 paciente de 71 años, con antecedentes de hta en tto, diabetes insulinodependiente, dislipidemia, hipotiroidismo en tto, enfermedad renal cronica etapa iii,tabaquismo cronico importante, en febrero de este año lo suspendio. Refiere que tiene principios de Alzheimer y parkinson?????? NO SALE REGISTRO DE DIAGNOSTICOS. Refiere que necesita ic a oftalmologo. Tiene astigmatismo y miopia, ocupa lentes para ambos trastornos de viciorefraccion, refiere que hace 4 meses que ve borroso utilizando lentes ópticos.  Fue operada hace mas de 2 años por retinopatia diabetica en ambos ojos. 

Al ex fisico: No observo ojo rojo. pupilas isocoricas y reactivas. no observo opacidades corneales. RFM presente. agudeza visual conservada.
\end{lstlisting}

En el archivo \texttt{cwlc.zip} del siquiente repositorio se encuentra publicado el conjunto de datos creado por el equipo de tecnologías de información. Las hipótesis diagnósticas están en los archivos con extensión \texttt{.txt}.

\begin{center}
	\url{https://doi.org/10.5281/zenodo.7555181}
\end{center}

\section*{Preguntas}

Responda las siguientes preguntas en un \textit{Jupyter Notebook} con código desarrollado en el lenguaje de programación Python.

\begin{enumerate}
	\item Importe cada una de las hipótesis diagnósticas y cuente cuántos archivos hay (este será su denominador en la prevalencia) y cuántos caracteres tiene cada sospecha diagnóstica.
	\item Con la utilización de expresiones regulares, detecte menciones de las siguientes enfermedades en las hipótesis diagnósticas. Recuerde que una enfermedad puede ser mencionada de distintas formas sinónimas. Cada una de las enfermedades tiene referenciado su código del vocabulario CIE-10 y un enlace a su entrada en Wikidata para que puedan ver más información sobre la enfermedad.
	      \begin{enumerate}
	      	\item[\href{https://www.wikidata.org/wiki/Q41861}{I10}] Hipertensión esencial (primaria)
	      	\item[\href{https://www.wikidata.org/wiki/Q3025883}{E11}] Diabetes mellitus tipo 2
	      	\item[\href{https://www.wikidata.org/wiki/Q128581}{C50}] Neoplasia maligna de mama
	      	\item[\href{https://www.wikidata.org/wiki/Q736715}{N18}] Enfermedad renal crónica 
	      	\item[\href{https://www.wikidata.org/wiki/Q11081}{G39}] Enfermedad de Alzheimer 
	      \end{enumerate}
	\item Tomando el conteo de menciones de cada una de las enfermedades y la cantidad de archivos analizado, calcule la prevalencia de cada una de las enfermedades en el conjunto de datos.
\end{enumerate}

\end{document}