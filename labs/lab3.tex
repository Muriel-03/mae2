\documentclass{article}

\usepackage[utf8]{inputenc}
\usepackage[spanish]{babel}
\usepackage{hyperref}
\usepackage{listings}
\renewcommand{\lstlistingname}{Texto}
\lstset{
    inputencoding = utf8,  % Input encoding
    extendedchars = true,  % Extended ASCII
    literate      =        % Support additional characters
      {á}{{\'a}}1  {é}{{\'e}}1  {í}{{\'i}}1 {ó}{{\'o}}1  {ú}{{\'u}}1
      {Á}{{\'A}}1  {É}{{\'E}}1  {Í}{{\'I}}1 {Ó}{{\'O}}1  {Ú}{{\'U}}1
      {à}{{\`a}}1  {è}{{\`e}}1  {ì}{{\`i}}1 {ò}{{\`o}}1  {ù}{{\`u}}1
      {À}{{\`A}}1  {È}{{\`E}}1  {Ì}{{\`I}}1 {Ò}{{\`O}}1  {Ù}{{\`U}}1
      {ä}{{\"a}}1  {ë}{{\"e}}1  {ï}{{\"i}}1 {ö}{{\"o}}1  {ü}{{\"u}}1
      {Ä}{{\"A}}1  {Ë}{{\"E}}1  {Ï}{{\"I}}1 {Ö}{{\"O}}1  {Ü}{{\"U}}1
      {â}{{\^a}}1  {ê}{{\^e}}1  {î}{{\^i}}1 {ô}{{\^o}}1  {û}{{\^u}}1
      {Â}{{\^A}}1  {Ê}{{\^E}}1  {Î}{{\^I}}1 {Ô}{{\^O}}1  {Û}{{\^U}}1
      {œ}{{\oe}}1  {Œ}{{\OE}}1  {æ}{{\ae}}1 {Æ}{{\AE}}1  {ß}{{\ss}}1
      {ẞ}{{\SS}}1  {ç}{{\c{c}}}1 {Ç}{{\c{C}}}1 {ø}{{\o}}1  {Ø}{{\O}}1
      {å}{{\aa}}1  {Å}{{\AA}}1  {ã}{{\~a}}1  {õ}{{\~o}}1 {Ã}{{\~A}}1
      {Õ}{{\~O}}1  {ñ}{{\~n}}1  {Ñ}{{\~N}}1  {¿}{{?`}}1  {¡}{{!`}}1
      {°}{{\textdegree}}1 {º}{{\textordmasculine}}1 {ª}{{\textordfeminine}}1
      {£}{{\pounds}}1  {©}{{\copyright}}1  {®}{{\textregistered}}1
      {«}{{\guillemotleft}}1  {»}{{\guillemotright}}1  {Ð}{{\DH}}1  {ð}{{\dh}}1
      {Ý}{{\'Y}}1    {ý}{{\'y}}1    {Þ}{{\TH}}1    {þ}{{\th}}1    {Ă}{{\u{A}}}1
      {ă}{{\u{a}}}1  {Ą}{{\k{A}}}1  {ą}{{\k{a}}}1  {Ć}{{\'C}}1    {ć}{{\'c}}1
      {Č}{{\v{C}}}1  {č}{{\v{c}}}1  {Ď}{{\v{D}}}1  {ď}{{\v{d}}}1  {Đ}{{\DJ}}1
      {đ}{{\dj}}1    {Ė}{{\.{E}}}1  {ė}{{\.{e}}}1  {Ę}{{\k{E}}}1  {ę}{{\k{e}}}1
      {Ě}{{\v{E}}}1  {ě}{{\v{e}}}1  {Ğ}{{\u{G}}}1  {ğ}{{\u{g}}}1  {Ĩ}{{\~I}}1
      {ĩ}{{\~\i}}1   {Į}{{\k{I}}}1  {į}{{\k{i}}}1  {İ}{{\.{I}}}1  {ı}{{\i}}1
      {Ĺ}{{\'L}}1    {ĺ}{{\'l}}1    {Ľ}{{\v{L}}}1  {ľ}{{\v{l}}}1  {Ł}{{\L{}}}1
      {ł}{{\l{}}}1   {Ń}{{\'N}}1    {ń}{{\'n}}1    {Ň}{{\v{N}}}1  {ň}{{\v{n}}}1
      {Ő}{{\H{O}}}1  {ő}{{\H{o}}}1  {Ŕ}{{\'{R}}}1  {ŕ}{{\'{r}}}1  {Ř}{{\v{R}}}1
      {ř}{{\v{r}}}1  {Ś}{{\'S}}1    {ś}{{\'s}}1    {Ş}{{\c{S}}}1  {ş}{{\c{s}}}1
      {Š}{{\v{S}}}1  {š}{{\v{s}}}1  {Ť}{{\v{T}}}1  {ť}{{\v{t}}}1  {Ũ}{{\~U}}1
      {ũ}{{\~u}}1    {Ū}{{\={U}}}1  {ū}{{\={u}}}1  {Ů}{{\r{U}}}1  {ů}{{\r{u}}}1
      {Ű}{{\H{U}}}1  {ű}{{\H{u}}}1  {Ų}{{\k{U}}}1  {ų}{{\k{u}}}1  {Ź}{{\'Z}}1
      {ź}{{\'z}}1    {Ż}{{\.Z}}1    {ż}{{\.z}}1    {Ž}{{\v{Z}}}1
      % ¿ and ¡ are not correctly displayed if inconsolata font is used
      % together with the lstlisting environment. Consider typing code in
      % external files and using \lstinputlisting to display them instead.      
  }

\title{Exploración de texto}
\author{Fabián Villena y Felipe Arias}
\date{Junio 2023}

\begin{document}

\maketitle

Existen múltiples tipos de narrativas dentro dentro del dominio clínico que pueden variar desde textos completamente ruidosos como el texto generado por profesionales de la salud en su práctica clínica hasta discursos casi perfectos en la escritura científica contenida en publicaciones de revistas.

Usted tendrá que verificar objetivamente estas diferencias entre dos \textit{corpora} de textos; el \textit{Corpus} de la Lista de espera que contiene sospechas diagnósticas de interconsultas a especialista y el \textit{Corpus} Biomédico que contiene texto de publicaciones de revistas científicas del dominio médico.

\noindent\begin{minipage}{.45\textwidth}
\begin{lstlisting}[breaklines=true, extendedchars=true,numbers=left,frame=single,caption=\textit{Corpus} de la lista de espera]
EPOCRefiere se ha mantenido estable, en Regulares condiciones generales, con aumento significativo en su disnea, ahora clase funcional III, sin angina u otros de alarma. Al examen: - BEG, normotensa, normocárdica, eupneica- RR2TSS- MP+ crepitos basales bilaterales- ABD: BDI, RHA+, sin masas palpables ni signos de irritacion peritoneal- Pulsos pedios  +/+, sin edema
\end{lstlisting}
\end{minipage}\hfill
\begin{minipage}{.45\textwidth}
	\begin{lstlisting}[breaklines=true, extendedchars=true,numbers=left,frame=single,caption=\textit{Corpus} Biomédico]
Los registros clínicos existen para documentar los síntomas iniciales del paciente, diagnósticos, medicamentos, tratamientos y resultados de estos tratamientos, teniendo además un carácter legal1. La información contenida en estos registros puede ser clasificada en estructurada y no estructurada.
	\end{lstlisting}
\end{minipage}

Los corpora preprocesados (\textit{tokenizados} y normalizados) se encuentran disponibles en las siguientes direcciones:

\begin{center}
	\url{https://dcc.uchile.cl/~fvillena/files/biomedical_corpus.txt}
\end{center}

\begin{center}
	\url{https://dcc.uchile.cl/~fvillena/files/waiting_list_corpus.txt}
\end{center}

\section*{Preguntas}

Responda las siguientes preguntas en un \textit{Jupyter Notebook} con código desarrollado en el lenguaje de programación Python.

\begin{enumerate}
	\item Desarrolle funciones para calcular el tamaño del \textit{corpus}, el tamaño del vocabulario, la diversidad léxica y la frecuencia de cada palabra del vocabulario y aplíquelas en cada \textit{corpus}.
	\item Construya un gráfico para cada \textit{corpus} que muestre en el eje x las palabras del vocabulario y en el eje y la frecuencia de cada una de las palabras.
	\item Discuta las diferencias que usted encontró entre ambos \textit{corpora} tomando en cuenta las métricas calculadas, el vocabulatio y la distribución de las palabras en los \textit{corpora}.
	\item Construya un Wordcloud para cada uno de los \textit{corpora}.
\end{enumerate}

\end{document}
